\documentclass{article}
\usepackage{fancyhdr}
\usepackage{lmodern}
\usepackage{fontspec}
\usepackage{polyglossia}
\usepackage{poemscol}
\usepackage{graphicx}
\setmainfont[Script=Kannada,Path=./]{lohit.ttf}
\begin{document}
\begin{poem}
\sequencefirstsectiontitle{ಬಣ್ಣದ ತಗಡಿನ ತುತ್ತೂರಿ (ಜಿ ಪಿ ರಾಜರತ್ನಂ  
)}
\begin{stanza}
ಬಣ್ಣದ ತಗಡಿನ ತುತ್ತೂರಿ \verseline
ಕಾಸಿಗೆ ಕೊಂಡನು ಕಸ್ತೂರಿ \verseline
ಸರಿಗಮ ಪದನಿಸ ಊದಿದನು \verseline
ಸನಿದಪ ಮಗರಿಸ ಊದಿದನು
\end{stanza} 
\begin{stanza}
ತನಗೇ ತುತ್ತೂರಿ ಇದೆ ಎಂದು \verseline
ಬೇರಾರಿಗೂ ಅದು ಇಲ್ಲೆಂದ \verseline
ಕಸ್ತೂರಿ ನಡೆದನು ಬೀದಿಯಲಿ \verseline
ಜಂಭದ ಕೋಳಿಯ ರೀತಿಯಲಿ
\end{stanza}
\begin{figure}[h!]
    \centering
    \includegraphics[scale=0.3]{BannadaTagadinaTuttoori.png}
\end{figure}
\begin{stanza}
ತುತ್ತೂರಿ ಊದುತ ಕೊಳದ ಬಳಿ \verseline
ನಡೆದನು ಕಸ್ತೂರಿ ಸಂಜೆಯಲಿ \verseline
ಜಾರಿತು ನೀರಿಗೆ ತುತ್ತೂರಿ \verseline
ಗಂಟಲು ಕಟ್ಟಿತು ನೀರೂರಿ
\end{stanza}
\begin{stanza}
ಸರಿಗಮ ಊದಲು ನೋಡಿದನು \verseline
ಗಗಗಗ ಸದ್ದನು ಮಾಡಿದನು \verseline
ಬಣ್ಣವು ನೀರಿನ ಪಾಲಾಯ್ತು \verseline
ಬಣ್ಣದ ತುತ್ತೂರಿ ಬೋಳಾಯ್ತು
\end{stanza}
\begin{stanza}
ತುತ್ತೂರಿ ಬಣ್ಣವು ಹಾಳಾಯ್ತು \verseline
ಜಂಭದ ಕೋಳಿಗೆ ಗೋಳಾಯ್ತು
\end{stanza}
\end{poem}
\end{document}