\documentclass[a5paper]{article}
\usepackage{fancyhdr}
\usepackage{lmodern}
\usepackage{fontspec}
\usepackage{polyglossia}
\usepackage{poemscol}
\usepackage{graphicx}
\usepackage{geometry}
\usepackage{eso-pic}
\setmainfont[Script=Kannada,Path=./]{lohit.ttf}
% Redefine \poemlines to set line numbers to zero
% Define custom page numbering
\pagestyle{fancy}
\fancyhf{}
\rhead{\thepage} % Page number on the right
\lhead{} % No page number on the left
\renewcommand{\headrulewidth}{0pt} % Remove header line
\begin{document}
{\Huge ಬಣ್ಣದ ತಗಡಿನ ತುತ್ತೂರಿ}
\large
\begin{poem}
  \raggedleft
  \begin{stanza}
   ಬಣ್ಣದ ತಗಡಿನ ತುತ್ತೂರಿ \verseline
   ಕಾಸಿಗೆ ಕೊಂಡನು ಕಸ್ತೂರಿ \verseline
   ಸರಿಗಮ ಪದನಿಸ ಊದಿದನು \verseline
   ಸನಿದಪ ಮಗರಿಸ ಊದಿದನು
  \end{stanza}
  \raggedleft
  \begin{stanza}
    ತನಗೇ ತುತ್ತೂರಿ ಇದೆ ಎಂದು \verseline
    ಬೇರಾರಿಗೂ ಅದು ಇಲ್ಲೆಂದ \verseline
    ಕಸ್ತೂರಿ ನಡೆದನು ಬೀದಿಯಲಿ \verseline
    ಜಂಭದ ಕೋಳಿಯ ರೀತಿಯಲಿ
  \end{stanza}
  \raggedleft
  \begin{stanza}
    ತುತ್ತೂರಿ ಊದುತ ಕೊಳದ ಬಳಿ \verseline
    ನಡೆದನು ಕಸ್ತೂರಿ ಸಂಜೆಯಲಿ \verseline
    ಜಾರಿತು ನೀರಿಗೆ ತುತ್ತೂರಿ \verseline
    ಗಂಟಲು ಕಟ್ಟಿತು ನೀರೂರಿ
  \end{stanza}
  \raggedleft
  \begin{stanza}
    ಸರಿಗಮ ಊದಲು ನೋಡಿದನು \verseline
    ಗಗಗಗ ಸದ್ದನು ಮಾಡಿದನು \verseline
    ಬಣ್ಣವು ನೀರಿನ ಪಾಲಾಯ್ತು \verseline
    ಬಣ್ಣದ ತುತ್ತೂರಿ ಬೋಳಾಯ್ತು
  \end{stanza}
  \raggedleft
  \begin{stanza}
    ತುತ್ತೂರಿ ಬಣ್ಣವು ಹಾಳಾಯ್ತು \verseline
    ಜಂಭದ ಕೋಳಿಗೆ ಗೋಳಾಯ್ತು
  \end{stanza}
\end{poem}
\raggedleft
ಜಿ ಪಿ ರಾಜರತ್ನಂ
% Add the image as an overlay
\AddToShipoutPicture*{
  \put(00,200){\includegraphics[width=6.5cm]{boy.png}} % Adjust the coordinates and size as needed
}
\end{document}
